\thispagestyle{empty}

\newpage
\thispagestyle{empty}
\vspace*{3.5cm}
\begin{center}
{\LARGE \bf Engenharia de Software Moderna}\\ 
\vspace*{0.8cm}
{\Huge \bf Coletânea de Artigos Didáticos}\\ 
\vspace*{3cm}
{\Large \bf Marco Tulio Valente}
\end{center}
\newpage

\thispagestyle{empty}
\vspace*{3cm}
\begin{center}
{\Large  Versão 0.1 - janeiro 2023}\\ 
\vspace*{1cm}
{Direitos autorais reservados. Esta versão é para uso pessoal e individual, sendo proibida qualquer forma de redistribuição.}
\end{center}
\newpage

\tableofcontents
